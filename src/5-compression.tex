\section{E-Commerce and Market Intelligence}
In addition new impacts and development directions in business intelligence and analytics big data for e-commerce organizations.

The excitement surrounding BI\&A and Big Data has arguably
been generated primarily from the web and e-commerce
communities. Significant market transformation has been
accomplished by leading e-commerce vendors such Amazon
and eBay through their innovative and highly scalable ecommerce
platforms and product recommender systems.
Major Internet firms such as Google, Amazon, and Facebook
continue to lead the development of web analytics, cloud
computing, and social media platforms. The emergence of
customer-generated Web 2.0 content on various forums,
newsgroups, social media platforms, and crowd-sourcing
systems offers another opportunity for researchers and practitioners to “listen” to the voice of the market from a vast
number of business constituents that includes customers, employees,
investors, and the media (Doan et al. 2011; O’Rielly
2005). Unlike traditional transaction records collected from
various legacy systems of the 1980s, the data that e-commerce
systems collect from the web are less structured and often
contain rich customer opinion and behavioral information.
For social media analytics of customer opinions, text analysis
and sentiment analysis techniques are frequently adopted
(Pang and Lee 2008). Various analytical techniques have also
been developed for product recommender systems, such as
association rule mining, database segmentation and clustering,
anomaly detection, and graph mining (Adomavicius and
Tuzhilin 2005). Long-tail marketing accomplished by
reaching the millions of niche markets at the shallow end of
the product bitstream has become possible via highly targeted
searches and personalized recommendations (Anderson
2004).
The Netfix Prize competition for the best collaborative
filtering algorithm to predict user movie ratings helped generate
significant academic and industry interest in recommender
systems development and resulted in awarding the grand prize
of \$1 million to the Bellkor’s Pragmatic Chaos team, which
surpassed Netflix’s own algorithm for predicting ratings by
10.06 percent. However, the publicity associated with the
competition also raised major unintended customer privacy
concerns.

In conclusion much BI\&A-related e-commerce research and development
information is appearing in academic IS and CS papers as
well as in popular IT magazines.