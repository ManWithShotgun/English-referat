\section{BI\&A 2.0}
The Internet and the Web began to offer
unique data collection and analytical research and development opportunities. In 2000s the HTTP-based Web 1.0 systems,
characterized by Web search engines such as Google and
Yahoo and e-commerce businesses such as Amazon and
eBay, allow organizations to present their businesses online
and interact with their customers directly. 
In addition to porting their traditional RDBMS-based product information
and business contents online, detailed and IP-specific user
search and interaction logs that are collected seamlessly
through cookies and server logs have become a new gold
mine for understanding customers’ needs and identifying new
business opportunities. Web intelligence, web analytics, and
the user-generated content collected through Web 2.0-based
social and crowd-sourcing systems (Doan et al. 2011;
O’Reilly 2005) have ushered in a new and exciting era of
BI\&A 2.0 research in the 2000s, centered on text and web
analytics for unstructured web contents.

An immense amount of company, industry, product, and
customer information can be gathered from the web and
organized and visualized through various text and web mining
techniques. By analyzing customer clickstream data logs,
web analytics tools such as Google Analytics can provide a
trail of the user’s online activities and reveal the user’s
browsing and purchasing patterns. Web site design, product
placement optimization, customer transaction analysis, market
structure analysis, and product recommendations can be
accomplished through web analytics. The many Web 2.0
applications developed after 2004 have also created an abundance of user-generated content from various online social
media such as forums, online groups, web blogs, social networking sites, social multimedia sites (for photos and videos),
and even virtual worlds and social games (O’Reilly 2005). In
addition to capturing celebrity chatter, references to everyday
events, and socio-political sentiments expressed in these
media, Web 2.0 applications can efficiently gather a large
volume of timely feedback and opinions from a diverse
customer population for different types of businesses.

Many marketing researchers believe that social media
analytics presents a unique opportunity for businesses to treat
the market as a “conversation” between businesses and
customers instead of the traditional business-to-customer,
one-way “marketing” (Lusch et al. 2010). Unlike BI\&A 1.0
technologies that are already integrated into commercial
enterprise IT systems, future BI\&A 2.0 systems will require
the integration of mature and scalable techniques in text
mining (e.g., information extraction, topic identification,
opinion mining, question-answering), web mining, social
network analysis, and spatial-temporal analysis with existing
DBMS-based BI\&A 1.0 systems.

Except for basic query and search capabilities, no advanced
text analytics for unstructured content are currently considered in the 13 capabilities of the Gartner BI platforms.
Several, however, are listed in the Gartner BI Hype Cycle,
including information semantic services, natural language
question answering, and content/text analytics (Bitterer 2011). 
New IS and CS courses in text mining and web mining have
emerged to address needed technical training.

In conclusion it can be notified that in 2000s, BI\&A created big jump in infrastructure natural language, information extraction, topic identification, opinion mining, question-answering. And created new directions in IT.
