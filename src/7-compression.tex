\section{Science and Technology}

This chapter describes using business intelligence and analytics big data for science. Describes tools and algorithms for successful opening of new achievements and researchs.

Many areas of science and technology (S\&T) are reaping the
benefits of high-throughput sensors and instruments, from
astrophysics and oceanography, to genomics and environmental
research. To facilitate information sharing and data
analytics, the National Science Foundation (NSF) recently
mandated that every project is required to provide a data
management plan. Cyber-infrastructure, in particular, has
become critical for supporting such data-sharing initiatives.

The 2012 NSF BIGDATA program solicitation is an obvious
example of the U.S. government funding agency’s concerted
efforts to promote big data analytics. The program
aims to advance the core scientific and technological
means of managing, analyzing, visualizing, and extracting
useful information from large, diverse, distributed
and heterogeneous data sets so as to accelerate
the progress of scientific discovery and innovation;
lead to new fields of inquiry that would not
otherwise be possible; encourage the development of
new data analytic tools and algorithms; facilitate
scalable, accessible, and sustainable data infrastructure;
increase understanding of human and social
processes and interactions; and promote economic
growth and improved health and quality of life.

Several S\&T disciplines have already begun their journey
toward big data analytics. For example, in biology, the NSF
funded iPlant Collaborative is using cyberinfrastructure to
support a community of researchers, educators, and students
working in plant sciences. iPlant is intended to foster a new
generation of biologists equipped to harness rapidly expanding
computational techniques and growing data sets to
address the grand challenges of plant biology. The iPlant data
set is diverse and includes canonical or reference data,
experimental data, simulation and model data, observational
data, and other derived data. It also offers various open
source data processing and analytics tools.

In astronomy, the Sloan Digital Sky Survey (SDSS) shows
how computational methods and big data can support and
facilitate sense making and decision making at both the
macroscopic and the microscopic level in a rapidly growing
and globalized research field. The SDSS is one of the most
ambitious and influential surveys in the history of astronomy.
Over its eight years of operation, it has obtained deep, multicolor
images covering more than a quarter of the sky and
created three-dimensional maps containing more than 930,000
galaxies and over 120,000 quasars. Continuing to gather data
at a rate of 200 gigabytes per night, SDSS has amassed more
than 140 terabytes of data. The international Large Hadron
Collider (LHC) effort for high-energy physics is another
example of big data, producing about 13 petabytes of data in
a year (Brumfiel 2011).

In conclusion it can be notified that business intelligence and analytics big data contributed to the big jump in astronomy, physics  high-energy. The relationship between BI\&A and science contributed to delevopment of new data processing and analytics tools, algorithms.