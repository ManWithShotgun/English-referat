AJAX -- подход к построению интерактивных пользовательских интерфейсов веб-приложений, заключающийся в «фоновом» обмене данными браузера с веб-сервером. В результате, при обновлении данных веб-страница не перезагружается полностью, и веб-приложения становятся быстрее и удобнее.

API(application programming interface) -- набор готовых классов, процедур, функций, структур и констант, предоставляемых приложением (библиотекой, сервисом) или операционной системой для использования во внешних программных продуктах. Используется программистами при написании всевозможных приложений.

API GData -- Google Data Protocol, протокол для общения с серверами Google.

Business Intelligence and Analytics (BI\&A) -- Бизнес-анализ и аналитика

Communities of Interest (S\&T) -- комитет по интересам общества

COPLINK -- Программное обеспечение IBM облегчающие формирование унифицированного представления информации об общественной безопасности с помощью аналитических инструментов и средств, что облегчает командование, расследование, предупреждение и реагирование.

Dark Web -- В <<тёмной паутине>> находятся веб-страницы, не связанные с другими гиперссылками (например, тупиковые веб-страницы, динамически создаваемые скриптами на самих сайтах, по запросу на которые не ведут прямые ссылки), а также сайты, доступ к которым открыт только для зарегистрированных пользователей и интернет-страницы, доступные только по паролю.

Department of Defense (DOD) -- Департамент обороны

Defense Advanced Research Project Agency (DARPA) -- Агентство перспективных исследований

Electronic health records (EHR) -- электронные медицинские записи

ETL (Extract, Transform, Load) --  один из основных процессов в управлении хранилищами данных, который включает в себя: извлечение данных из внешних источников; их трансформация и очистка, чтобы они соответствовали потребностям бизнес-модели; и загрузка их в хранилище данных.

Flickr --  фотохостинг, предназначенный для хранения и дальнейшего использования пользователем цифровых фотографий и видеороликов. Является одним из первых сервисов Web 2.0. Один из самых популярных сайтов для размещения фотографий.

Google App Engine -- служба хостинга сайтов и web-приложений на серверах Google с бесплатным названием либо с собственным названием, задействованным с помощью служб Google.

Google Bigtable -- проприетарная высокопроизводительная база данных, построенная на основе Google File System (GFS), Chubby Lock Service и некоторых других продуктах Google. 

Google Maps -- набор приложений, построенных на основе бесплатного картографического сервиса и технологии, предоставляемых компанией Google. Сервис представляет собой карту и спутниковые снимки Земли.

Google Translate --  веб-служба компании Google, предназначенная для автоматического перевода части текста или веб-страницы на другой язык. Для некоторых языков пользователям предлагаются варианты переводов, например, для технических терминов, которые должны быть в будущем включены в обновления системы перевода.

Hadoop -- проект фонда Apache Software Foundation, свободно распространяемый набор утилит, библиотек и фреймворк для разработки и выполнения распределённых программ, работающих на кластерах из сотен и тысяч узлов. Используется для реализации поисковых и контекстных механизмов многих высоконагруженных веб-сайтов, в том числе, для Yahoo и Facebook

Health Insurance Portability and Accountability Act (HIPAA) -- переносимость медицинского страхования и Закон о подотчетности.

HTTP -- протокол прикладного уровня передачи данных (изначально в виде гипертекстовых документов в формате «HTML», в настоящий момент используется для передачи произвольных данных).

IaaS -- одна из моделей обслуживания в облачных вычислениях, по которой потребителям предоставляются по подписке фундаментальные информационно-технологические ресурсы виртуальные серверы с заданной вычислительной мощностью, операционной системой (чаще всего предустановленной провайдером из шаблона) и доступом к сетям.

International Classification of Diseases codes (ICD-9) -- Коды международной классификации болезней

IBM -- американская компания со штаб-квартирой в Армонке (штат Нью-Йорк), один из крупнейших в мире производителей и поставщиков аппаратного и программного обеспечения, а также ИТ-сервисов и консалтинговых услуг.

Institutional Review Board (IRB) -- Комиссия по институциональному обзору

Java -- сильно типизированный объектно-ориентированный язык программирования, разработанный компанией Sun Microsystems (в последующем приобретённой компанией Oracle). Приложения Java обычно транслируются в специальный байт-код, поэтому они могут работать на любой компьютерной архитектуре, с помощью виртуальной Java-машины.

JavaScript -- мультипарадигменный язык программирования. Поддерживает объектно-ориентированный, императивный и функциональный стили. Является реализацией языка ECMAScript

JSON --  текстовый формат обмена данными, основанный на JavaScript. Как и многие другие текстовые форматы, JSON легко читается людьми.

Latent Dirichlet allocation (LDA) -- скрытое распределение Дирихле

Microsoft's Windows Azure -- название облачной платформы Microsoft. Предоставляет возможность разработки и выполнения приложений и хранения данных на серверах, расположенных в распределённых дата-центрах.

National Institutes of Health (NIH) -- Национальный институт здравоохранения

Natural language processing (NLP) -- обработка естественных языков

National Science Foundation (NSF) -- Национальный научный фонд

PaaS -- модель предоставления облачных вычислений, при которой потребитель получает доступ к использованию информационно-технологических платформ: операционных систем, систем управления базами данных, связующему программному обеспечению, средствам разработки и тестирования, размещённым у облачного провайдера.

Part-of-speech-tagging (POST) -- частота речи

Q/A (questions and answers) -- система вопрос-ответ для тестирования различных знаний.

REST -- архитектурный стиль взаимодействия компонентов распределённого приложения в сети. REST представляет собой согласованный набор ограничений, учитываемых при проектировании распределённой гипермедиа-системы.

SaaS -- Одна из форм облачных вычислений, модель обслуживания, при которой подписчикам предоставляется готовое прикладное программное обеспечение, полностью обслуживаемое провайдером.

Smart Health and Wellbeing (SHB) -- Здравоохранение и благополучие

SOA (service-oriented architecture) --  модульный подход к разработке программного обеспечения, основанный на использовании распределённых, слабо связанных заменяемых компонентов, оснащённых стандартизированными интерфейсами для взаимодействия по стандартизированным протоколам.

Symptom–disease–treatment (SDT) -- правило ассоциации с симптомами болезни

Геномики - Персональная геномика является разделом геномики, связанным с секвенированием и анализом генома человека. Стадия генотипирования использует различные методы, включая однонуклеотидно полиморфные (МНП) анализирующие чипы (как правило, составляющие 0,02 \% генома), а также частичное или полное секвенирование генома. После расшифровки генотипа его можно проанализировать при помощи опубликованной литературы для определения вероятности риска заболеваний.

Секвенирование биополимеров (белков и нуклеиновых кислот — ДНК и РНК) — определение их аминокислотной или нуклеотидной последовательности (от лат. sequentum — последовательность). В результате секвенирования получают формальное описание первичной структуры линейной макромолекулы в виде последовательности мономеров в текстовом виде. 
