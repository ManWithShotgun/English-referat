\section{E-Government and Politics 2.0 (Электронное правительство и политика 2.0)}

This chapter describes changes in e-commerce area with coming Web 2.0
and with coming new technologies in BI\&A research. The advent of Web 2.0 has generated much excitement for
reinventing governments. 

The 2008 U.S. House, Senate, and
presidential elections provided the first signs of success for
online campaigning and political participation. Dubbed
<<politics 2.0>>, politicians use the highly participatory and
multimedia web platforms for successful policy discussions,
campaign advertising, voter mobilization, event announcements,
and online donations. As government and political
processes become more transparent, participatory, online, and
multimedia-rich, there is a great opportunity for adopting
BI\&A research in e-government and politics 2.0 applications.
Selected opinion mining, social network analysis, and social
media analytics techniques can be used to support online
political participation, e-democracy, political blogs and
forums analysis, e-government service delivery, and process
transparency and accountability~\cite{Chen:2006}. For e-government applications, semantic information
directory and ontological development (as exemplified below)
can also be developed to better serve their target citizens.

Despite the significant transformational potential for BI\&A in
e-government research, there has been less academic research
than, for example, e-commerce-related BI\&A research. Egovernment
research often involves researchers from political
science and public policy. For example, Karpf (2009) analyzed
the growth of the political blogosphere in the United
States and found significant innovation of existing political
institutions in adopting blogging platforms into their Web
offerings. In his research, 2D blogspace mapping with composite
rankings helped reveal the partisan makeup of the
American political blogsphere. Yang and Callan (2009)
demonstrated the value for ontology development for government
services through their development of the OntoCop
system, which works interactively with a user to organize and
summarize online public comments from citizens.

In conclusion it can be notified that e-commerce area will grow with BI\&A science. Also it can be notified that more important for e-commerce will be the BI\&A in e-government research and e-commerce-related BI\&A research directions.