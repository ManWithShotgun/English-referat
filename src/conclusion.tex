Целью реферата было исследование преимуществ и недостатков ОС \textit{Linux}. Был поставлен ряд задач, которые необходимо было выполнить, для достижения намеченной цели. Если рассмотреть последовательно каждый пункт, то можно сделать вывод, что цель реферата достигнута: дан развернутый ответ на вопрос, что такое \textit{Linux}. Рассмотрена поэтапно история создания ОС \textit{Linux}. Проанализированы сильные и слабые строны современных ОС, а также выявлены основные преимущества и недостатки. Сделаны соответствующие выводы о перспективе развития \textit{Linux}.

Во второй главе реферата рассмотрен процесс рождения $Z^\prime$ бозонов в протон-протонных столкновениях с учётом эффектов $Z$-$Z^\prime$ смешивания. Рассмотрены интурменты библиотеки \textit{PYTHIA} для имитационного моделирования процессов взаимодействия элементарных частиц при высоких энергиях. Исследован процесс рождения $Z^\prime$-бозонов в процессе $pp \rightarrow Z^\prime \rightarrow l^+l^- + X$ с учетом эффектов $Z$--$Z^\prime$ смешивания.