\section{BI\&A 1.0}
In the first place the term intelligencehas been used by researchers in
artificial intelligence since the 1950s. Business intelligence
became a popular term in the business and IT communities
only in the 1990s. In the late 2000s, business analyticswas
introduced to represent the key analytical component in BI~\cite{davenport:2006}. More recently big dataand big data
analyticshave been used to describe the data sets and analytical techniques in applications that are so large (from
terabytes to exabytes) and complex (from sensor to social
media data) that they require advanced and unique data

As a data-centric approach, BI\&A has its roots in the longstanding database management field. It relies heavily on
various data collection, extraction, and analysis technologies
(Chaudhuri et al. 2011; Turban et al. 2008; Watson and
Wixom 2007). The BI\&A technologies and applications
currently adopted in industry can be considered as BI\&A 1.0,
where data are mostly structured, collected by companies
through various legacy systems, and often stored in commercial relational database management systems (RDBMS). The
analytical techniques commonly used in these systems,
popularized in the 1990s, are grounded mainly in statistical
methods developed in the 1970s and data mining techniques
developed in the 1980s.

Data management and warehousing is considered the foundation of BI\&A 1.0. Design of data marts and tools for
extraction, transformation, and load (ETL) are essential for
converting and integrating enterprise-specific data. Database
query, online analytical processing (OLAP), and reporting
tools based on intuitive, but simple, graphics are used to
explore important data characteristics. Business performance
management (BPM) using scorecards and dashboards help
analyze and visualize a variety of performance metrics. In
addition to these well-established business reporting functions, statistical analysis and data mining techniques are
adopted for association analysis, data segmentation and
clustering, classification and regression analysis, anomaly
detection, and predictive modeling in various business applications. Most of these data processing and analytical technologies have already been incorporated into the leading commercial BI platforms offered by major IT vendors including
Microsoft, IBM, Oracle, and SAP (Sallam et al. 2011).

Among the 13 capabilities considered essential for BI platforms, according to the Gartner report by Sallam et al. (2011),
the following eight are considered BI\&A 1.0: reporting,
dashboards, ad hocquery, search-based BI, OLAP, interactive
visualization, scorecards, predictive modeling, and data
mining. A few BI\&A 1.0 areas are still under active development based on the Gartner BI Hype Cycle analysis for
emerging BI technologies, which include data mining workbenchs, column-based DBMS, in-memory DBMS, and realtime decision tools (Bitterer 2011). Academic curricula in
Information Systems (IS) and Computer Science (CS) often include well-structured courses such as database management
systems, data mining, and multivariate statistics.

In conclusion the opportunities associated with data and analysis in different organizations have helped generate significant interest
in BI\&A, which is often referred to as the techniques, technologies, systems, practices, methodologies, and applications
that analyze critical business data to help an enterprise better
understand its business and market and make timely business
decisions. 