\section*{BI\&A 1.0}
As a data-centric approach, BI\&A has its roots in the longstanding database management field. It relies heavily on
various data collection, extraction, and analysis technologies
(Chaudhuri et al. 2011; Turban et al. 2008; Watson and
Wixom 2007). The BI\&A technologies and applications
currently adopted in industry can be considered as BI\&A 1.0,
where data are mostly structured, collected by companies
through various legacy systems, and often stored in commercial relational database management systems (RDBMS). The
analytical techniques commonly used in these systems,
popularized in the 1990s, are grounded mainly in statistical
methods developed in the 1970s and data mining techniques
developed in the 1980s. 
Data management and warehousing is considered the foundation of BI\&A 1.0. Design of data marts and tools for
extraction, transformation, and load (ETL) are essential for
converting and integrating enterprise-specific data. Database
query, online analytical processing (OLAP), and reporting
tools based on intuitive, but simple, graphics are used to
explore important data characteristics. Business performance
management (BPM) using scorecards and dashboards help
analyze and visualize a variety of performance metrics. In
addition to these well-established business reporting functions, statistical analysis and data mining techniques are
adopted for association analysis, data segmentation and
clustering, classification and regression analysis, anomaly
detection, and predictive modeling in various business applications. Most of these data processing and analytical technologies have already been incorporated into the leading commercial BI platforms offered by major IT vendors including
Microsoft, IBM, Oracle, and SAP (Sallam et al. 2011). 
Among the 13 capabilities considered essential for BI platforms, according to the Gartner report by Sallam et al. (2011),
the following eight are considered BI\&A 1.0: reporting,
dashboards, ad hocquery, search-based BI, OLAP, interactive
visualization, scorecards, predictive modeling, and data
mining. A few BI\&A 1.0 areas are still under active development based on the Gartner BI Hype Cycle analysis for
emerging BI technologies, which include data mining workbenchs, column-based DBMS, in-memory DBMS, and realtime decision tools (Bitterer 2011). Academic curricula in
Information Systems (IS) and Computer Science (CS) often include well-structured courses such as database management
systems, data mining, and multivariate statistics.

\section*{BI\&A 2.0}
Since the early 2000s, the Internet and the Web began to offer
unique data collection and analytical research and development opportunities. The HTTP-based Web 1.0 systems,
characterized by Web search engines such as Google and
Yahoo and e-commerce businesses such as Amazon and
eBay, allow organizations to present their businesses online
and interact with their customers directly. In addition to
porting their traditional RDBMS-based product information
and business contents online, detailed and IP-specific user
search and interaction logs that are collected seamlessly
through cookies and server logs have become a new gold
mine for understanding customers’ needs and identifying new
business opportunities. Web intelligence, web analytics, and
the user-generated content collected through Web 2.0-based
social and crowd-sourcing systems (Doan et al. 2011;
O’Reilly 2005) have ushered in a new and exciting era of
BI\&A 2.0 research in the 2000s, centered on text and web
analytics for unstructured web contents.
An immense amount of company, industry, product, and
customer information can be gathered from the web and
organized and visualized through various text and web mining
techniques. By analyzing customer clickstream data logs,
web analytics tools such as Google Analytics can provide a
trail of the user’s online activities and reveal the user’s
browsing and purchasing patterns. Web site design, product
placement optimization, customer transaction analysis, market
structure analysis, and product recommendations can be
accomplished through web analytics. The many Web 2.0
applications developed after 2004 have also created an abundance of user-generated content from various online social
media such as forums, online groups, web blogs, social networking sites, social multimedia sites (for photos and videos),
and even virtual worlds and social games (O’Reilly 2005). In
addition to capturing celebrity chatter, references to everyday
events, and socio-political sentiments expressed in these
media, Web 2.0 applications can efficiently gather a large
volume of timely feedback and opinions from a diverse
customer population for different types of businesses.
Many marketing researchers believe that social media
analytics presents a unique opportunity for businesses to treat
the market as a “conversation” between businesses and
customers instead of the traditional business-to-customer,
one-way “marketing” (Lusch et al. 2010). Unlike BI\&A 1.0
technologies that are already integrated into commercial
enterprise IT systems, future BI\&A 2.0 systems will require
the integration of mature and scalable techniques in text
mining (e.g., information extraction, topic identification,
opinion mining, question-answering), web mining, social
network analysis, and spatial-temporal analysis with existing
DBMS-based BI\&A 1.0 systems.
Except for basic query and search capabilities, no advanced
text analytics for unstructured content are currently considered in the 13 capabilities of the Gartner BI platforms.
Several, however, are listed in the Gartner BI Hype Cycle,
including information semantic services, natural language
question answering, and content/text analytics (Bitterer 2011). 
New IS and CS courses in text mining and web mining have
emerged to address needed technical training.

\section*{BI\&A 3.0}

Whereas web-based BI\&A 2.0 has attracted active research
from academia and industry, a new research opportunity in
BI\&A 3.0 is emerging. As reported prominently in an
October 2011 article in The Economist (2011), the number of
mobile phones and tablets (about 480 million units) surpassed
the number of laptops and PCs (about 380 million units) for
the first time in 2011. Although the number of PCs in use
surpassed 1 billion in 2008, the same article projected that the
number of mobile connected devices would reach 10 billion
in 2020. Mobile devices such as the iPad, iPhone, and other
smart phones and their complete ecosystems of downloadable
applicationss, from travel advisories to multi-player games,
are transforming different facets of society, from education to
healthcare and from entertainment to governments. Other
sensor-based Internet-enabled devices equipped with RFID,
barcodes, and radio tags (the “Internet of Things”) are
opening up exciting new steams of innovative applications.
The ability of such mobile and Internet-enabled devices to
support highly mobile, location-aware, person-centered, and
context-relevant operations and transactions will continue to
offer unique research challenges and opportunities throughout
the 2010s. Mobile interface, visualization, and HCI
(human–computer interaction) design are also promising
research areas. Although the coming of the Web 3.0 (mobile
and sensor-based) era seems certain, the underlying mobile
analytics and location and context-aware techniques for
collecting, processing, analyzing and visualizing such largescale
and fluid mobile and sensor data are still unknown.
No integrated, commercial BI\&A 3.0 systems are foreseen for
the near future. Most of the academic research on mobile BI
is still in an embryonic stage. Although not included in the
current BI platform core capabilities, mobile BI has been
included in the Gartner BI Hype Cycle analysis as one of the
new technologies that has the potential to disrupt the BI
market significantly (Bitterer 2011). The uncertainty associated
with BI\&A 3.0 presents another unique research
direction for the IS community.
Table 1 summarizes the key characteristics of BI\&A 1.0, 2.0,
and 3.0 in relation to the Gartner BI platforms core capabilities
and hype cycle.
The decade of the 2010s promises to be an exciting one for
high-impact BI\&A research and development for both industry
and academia. The business community and industry have
already taken important steps to adopt BI\&A for their needs.
The IS community faces unique challenges and opportunities
in making scientific and societal impacts that are relevant and
long-lasting (Chen 2011a). IS research and education programs
need to carefully evaluate future directions, curricula,
and action plans, from BI\&A 1.0 to 3.0.

\section*{BI\&A Applications: From Big Data to Big Impact}

Several global business and IT trends have helped shape past
and present BI\&A research directions. International travel,
high-speed network connections, global supply-chain, and
outsourcing have created a tremendous opportunity for IT
advancement, as predicted by Thomas Freeman in his seminal
book, The World is Flat (2005). In addition to ultra-fast
global IT connections, the development and deployment of
business-related data standards, electronic data interchange
(EDI) formats, and business databases and information
systems have greatly facilitated business data creation and
utilization. The development of the Internet in the 1970s and
the subsequent large-scale adoption of the World Wide Web
since the 1990s have increased business data generation and
collection speeds exponentially. Recently, the Big Data era
has quietly descended on many communities, from governments
and e-commerce to health organizations. With an
overwhelming amount of web-based, mobile, and sensorgenerated
data arriving at a terabyte and even exabyte scale
(The Economist 2010a, 2010b), new science, discovery, and
insights can be obtained from the highly detailed, contextualized,
and rich contents of relevance to any business or
organization.
In addition to being data driven, BI\&A is highly applied and
can leverage opportunities presented by the abundant data and
domain-specific analytics needed in many critical and highimpact
application areas. Several of these promising and
high-impact BI\&A applications are presented below, with a
discussion of the data and analytics characteristics, potential
impacts, and selected illustrative examples or studies: (1) ecommerce
and market intelligence, (2) e-government and
politics 2.0, (3) science and technology, (4) smart health and
well-being, and (5) security and public safety. By carefully
analyzing the application and data characteristics, researchers
and practitioners can then adopt or develop the appropriate
analytical techniques to derive the intended impact. In addition
to technical system implementation, significant business
or domain knowledge as well as effective communication
skills are needed for the successful completion of such BI\&A
projects. IS departments thus face unique opportunities and
challenges in developing integrated BI\&A research and
education programs for the new generation of data/analyticssavvy
and business-relevant students and professionals (Chen
2011a).

\section*{E-Commerce and Market Intelligence}

The excitement surrounding BI\&A and Big Data has arguably
been generated primarily from the web and e-commerce
communities. Significant market transformation has been
accomplished by leading e-commerce vendors such Amazon
and eBay through their innovative and highly scalable ecommerce
platforms and product recommender systems.
Major Internet firms such as Google, Amazon, and Facebook
continue to lead the development of web analytics, cloud
computing, and social media platforms. The emergence of
customer-generated Web 2.0 content on various forums,
newsgroups, social media platforms, and crowd-sourcing
systems offers another opportunity for researchers and practitioners to “listen” to the voice of the market from a vast
number of business constituents that includes customers, employees,
investors, and the media (Doan et al. 2011; O’Rielly
2005). Unlike traditional transaction records collected from
various legacy systems of the 1980s, the data that e-commerce
systems collect from the web are less structured and often
contain rich customer opinion and behavioral information.
For social media analytics of customer opinions, text analysis
and sentiment analysis techniques are frequently adopted
(Pang and Lee 2008). Various analytical techniques have also
been developed for product recommender systems, such as
association rule mining, database segmentation and clustering,
anomaly detection, and graph mining (Adomavicius and
Tuzhilin 2005). Long-tail marketing accomplished by
reaching the millions of niche markets at the shallow end of
the product bitstream has become possible via highly targeted
searches and personalized recommendations (Anderson
2004).
The Netfix Prize competition2
for the best collaborative
filtering algorithm to predict user movie ratings helped generate
significant academic and industry interest in recommender
systems development and resulted in awarding the grand prize
of \$1 million to the Bellkor’s Pragmatic Chaos team, which
surpassed Netflix’s own algorithm for predicting ratings by
10.06 percent. However, the publicity associated with the
competition also raised major unintended customer privacy
concerns.
Much BI\&A-related e-commerce research and development
information is appearing in academic IS and CS papers as
well as in popular IT magazines.

\section*{E-Government and Politics 2.0}

The advent of Web 2.0 has generated much excitement for
reinventing governments. The 2008 U.S. House, Senate, and
presidential elections provided the first signs of success for
online campaigning and political participation. Dubbed
“politics 2.0,” politicians use the highly participatory and
multimedia web platforms for successful policy discussions,
campaign advertising, voter mobilization, event announcements,
and online donations. As government and political
processes become more transparent, participatory, online, and
multimedia-rich, there is a great opportunity for adopting
BI\&A research in e-government and politics 2.0 applications.
Selected opinion mining, social network analysis, and social
media analytics techniques can be used to support online
political participation, e-democracy, political blogs and
forums analysis, e-government service delivery, and process
transparency and accountability (Chen 2009; Chen et al.
2007). For e-government applications, semantic information
directory and ontological development (as exemplified below)
can also be developed to better serve their target citizens.
Despite the significant transformational potential for BI\&A in
e-government research, there has been less academic research
than, for example, e-commerce-related BI\&A research. Egovernment
research often involves researchers from political
science and public policy. For example, Karpf (2009) analyzed
the growth of the political blogosphere in the United
States and found significant innovation of existing political
institutions in adopting blogging platforms into their Web
offerings. In his research, 2D blogspace mapping with composite
rankings helped reveal the partisan makeup of the
American political blogsphere. Yang and Callan (2009)
demonstrated the value for ontology development for government
services through their development of the OntoCop
system, which works interactively with a user to organize and
summarize online public comments from citizens.

\section*{Science and Technology}

Many areas of science and technology (S\&T) are reaping the
benefits of high-throughput sensors and instruments, from
astrophysics and oceanography, to genomics and environmental
research. To facilitate information sharing and data
analytics, the National Science Foundation (NSF) recently
mandated that every project is required to provide a data
management plan. Cyber-infrastructure, in particular, has
become critical for supporting such data-sharing initiatives.
The 2012 NSF BIGDATA3
program solicitation is an obvious
example of the U.S. government funding agency’s concerted
efforts to promote big data analytics. The program
aims to advance the core scientific and technological
means of managing, analyzing, visualizing, and extracting
useful information from large, diverse, distributed
and heterogeneous data sets so as to accelerate
the progress of scientific discovery and innovation;
lead to new fields of inquiry that would not
otherwise be possible; encourage the development of
new data analytic tools and algorithms; facilitate
scalable, accessible, and sustainable data infrastructure;
increase understanding of human and social
processes and interactions; and promote economic
growth and improved health and quality of life.
Several S\&T disciplines have already begun their journey
toward big data analytics. For example, in biology, the NSF
funded iPlant Collaborative4
is using cyberinfrastructure to
support a community of researchers, educators, and students
working in plant sciences. iPlant is intended to foster a new
generation of biologists equipped to harness rapidly expanding
computational techniques and growing data sets to
address the grand challenges of plant biology. The iPlant data
set is diverse and includes canonical or reference data,
experimental data, simulation and model data, observational
data, and other derived data. It also offers various open
source data processing and analytics tools.
In astronomy, the Sloan Digital Sky Survey (SDSS)5
shows
how computational methods and big data can support and
facilitate sense making and decision making at both the
macroscopic and the microscopic level in a rapidly growing
and globalized research field. The SDSS is one of the most
ambitious and influential surveys in the history of astronomy.
Over its eight years of operation, it has obtained deep, multicolor
images covering more than a quarter of the sky and
created three-dimensional maps containing more than 930,000
galaxies and over 120,000 quasars. Continuing to gather data
at a rate of 200 gigabytes per night, SDSS has amassed more
than 140 terabytes of data. The international Large Hadron
Collider (LHC) effort for high-energy physics is another
example of big data, producing about 13 petabytes of data in
a year (Brumfiel 2011).
