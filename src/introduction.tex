Бизнес-анализ и аналитика (BI\&A) и связанные с ними
поле аналитики больших данных становится все более важным
как в академических, так и в деловых сферах
последние два десятилетия. Отраслевые исследования подчеркнули это
значительное развитие. Например, на основе опроса
более 4000 специалистов в области информационных технологий (ИТ) из 93
стран и 25 отраслей промышленности, отчет IBM Tech Trends
(2011) определили бизнес-аналитику как одну из четырех основных
технологических тенденций в 2010-м. В обзоре состояния
бизнес-аналитика от Bloomberg Businessweek (2011), 97 процентов компаний с доходом, превышающим 100 миллионов долларов США,
как было установлено, используют некоторую форму бизнес-аналитики. Отчет
Глобальным институтом McKinsey (Manyika et al., 2011) предсказано
что к 2018 году только Соединенные Штаты столкнутся с нехваткой
от 140 000 до 190 000 человек с глубокими аналитическими навыками,
а также дефицит 1,5 млн. менеджеров ориентированных на данные, с
новыми идеями для анализа больших данных для принятия эффективных решений.
Возможности, связанные с данными и анализом в разных
организациях помогли создать значительный интерес
в BI\&A, который часто называют техникой, технологиями,
системами, методами, методологиями и приложениями
которые анализируют критически важные бизнес-данные, чтобы помочь предприятию лучше
понимать свой бизнес и рынок и своевременно внести бизнес
решения. В дополнение к базовой обработке данных и
аналитические технологий, BI\&A включает бизнес-ориентированные
практики и методологии, которые могут применяться к различным
сферам как электронная коммерция, рыночная разведка,
электронное правительство, здравоохранение и безопасность.