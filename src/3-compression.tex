\section{BI\&A 3.0 (Бизнес-анализ и аналитика версия 3.0)}
This chapter describes most of the academic research on mobile BI, opening up exciting new steams of innovative applications and describes business intelligence and analytics in Web 3.0 area.

Whereas web-based BI\&A 2.0 has attracted active research
from academia and industry, a new research opportunity in
BI\&A 3.0 is emerging. As reported prominently in an
October 2011 article in The Economist (2011), the number of
mobile phones and tablets (about 480 million units) surpassed
the number of laptops and PCs (about 380 million units) for
the first time in 2011. Although the number of PCs in use
surpassed 1 billion in 2008, the same article projected that the
number of mobile connected devices would reach 10 billion
in 2020. Mobile devices such as the iPad, iPhone, and other
smart phones and their complete ecosystems of downloadable
applicationss, from travel advisories to multi-player games,
are transforming different facets of society, from education to
healthcare and from entertainment to governments. Other
sensor-based Internet-enabled devices equipped with RFID,
barcodes, and radio tags (the “Internet of Things”) are
opening up exciting new steams of innovative applications.
The ability of such mobile and Internet-enabled devices to
support highly mobile, location-aware, person-centered, and
context-relevant operations and transactions will continue to
offer unique research challenges and opportunities throughout
the 2010s. Mobile interface, visualization, and HCI
(human–computer interaction) design are also promising
research areas. Although the coming of the Web 3.0 (mobile
and sensor-based) era seems certain, the underlying mobile
analytics and location and context-aware techniques for
collecting, processing, analyzing and visualizing such largescale
and fluid mobile and sensor data are still unknown.

No integrated, commercial BI\&A 3.0 systems are foreseen for
the near future. Most of the academic research on mobile BI
is still in an embryonic stage. Although not included in the
current BI platform core capabilities, mobile BI has been
included in the Gartner BI Hype Cycle analysis as one of the
new technologies that has the potential to disrupt the BI
market significantly~\cite{Blei:2012}. The uncertainty associated
with BI\&A 3.0 presents another unique research
direction for the IS community.
Table 1 summarizes the key characteristics of BI\&A 1.0, 2.0,
and 3.0 in relation to the Gartner BI platforms core capabilities
and hype cycle.

In conclusion it can be notified that the decade of the 2010s was an exciting one for high-impact BI\&A research and development for both industry and academia. IS research and education programs
need to carefully evaluate future directions, curricula,
and action plans, from BI\&A 1.0 to 3.0. The business community and industry have already taken important steps to adopt BI\&A for their needs. The IS community faces unique challenges and opportunities
in making scientific and societal impacts that are relevant and
long-lasting~\cite{Chen:2006}. 